\documentclass[compress,xcolor=table]{beamer}

\usepackage{beamerstyle}

% --------------------------------------------------------------------------------------------------
%  initialize
% --------------------------------------------------------------------------------------------------

\begin{document}

\title{\texttt{eva4s}\\Evolutionary Algorithms for Scala}
\subtitle{Benchmarking Evolutionary Equation Solvers}
\author[Foken,Krause]{Nils Foken\\Christian Krause}
\institute[BA Leipzig]{Berufsakademie Leipzig}
\subject{Evolutionary Algorithms}
\keywords{EA}

\frame[plain]{\titlepage}

\begin{frame}
  \frametitle{Outline}
  \tableofcontents[hideallsubsections]
\end{frame}

% --------------------------------------------------------------------------------------------------
%  introduction
% --------------------------------------------------------------------------------------------------

%\section{Introduction}
%\subsection*{}

%\begin{frame}
%  \frametitle{Purpose}
%  \begin{itemize}
%    \item 
%  \end{itemize}
%\end{frame}

% --------------------------------------------------------------------------------------------------
%  main parts
% --------------------------------------------------------------------------------------------------

% show outline at the beginning of each section
\AtBeginSection[] {
  \begin{frame}<beamer>
    \frametitle{Outline}
    \tableofcontents[sectionstyle=show/shaded,subsectionstyle=show/show/hide]
  \end{frame}
}

% --------------------------------------------------------------------------------------------------
%  new section
% --------------------------------------------------------------------------------------------------

\section{\texttt{eva4s} \textendash\ Architecture}

\subsection{The Main Parts}

\begin{frame}
  \frametitle{The Main Parts (I) \textendash\ \texttt{Evolutionary[G,P]}}
  \begin{itemize}
    \item type parameterized
      \begin{description}
        \item[G] genome type
        \item[P] problem type
      \end{description}
    \item provides main functions
      \begin{description}
        \item[\texttt{val problem: P}]
        \item[\texttt{def ancestor: G}]
        \item[\texttt{def fitness(genome: G): Double}]
        \item[\texttt{def mutate(genome: G): G}]
        \item[\texttt{def recombine(p1: G, p2: G): Seq[G]}]
      \end{description}
  \end{itemize}
\end{frame}

\begin{frame}
  \frametitle{The Main Parts (II) \textendash\ \texttt{Evolver}}
  \begin{itemize}
    \item executes an \texttt{Evolutionary}
    \item accepts parameters like
    \begin{itemize}
      \item generations
      \item population size
      \item environmental selection
      \item parental selection
    \end{itemize}
  \end{itemize}
\end{frame}

\subsection{Special Parameters}

\begin{frame}
  \frametitle{\texttt{Selector}}
  \framesubtitle{\texttt{(Seq[Individual[G]],Seq[Individual[G]]) => Seq[Individual[G]]}}
  \begin{itemize}
    \item models environmental selection
    \item determines how the individuals for the next generation are chosen
    \item implementations:
    \begin{description}
      \item[\texttt{ChildSelection}]
      \item[\texttt{CommaSelection}]
      \item[\texttt{PlusSelection}]
      \item[\texttt{RandomSelection}]
    \end{description}
  \end{itemize}
\end{frame}

\begin{frame}
  \frametitle{\texttt{Matchmaker}}
  \framesubtitle{\texttt{(Seq[Individual[G]],Int) => Seq[Pair[Individual[G],Individual[G]]]}}
  \begin{itemize}
    \item models parental selection
    \item essentially pairs individuals up with each other
    \item implementations:
    \begin{description}
      \item[\texttt{RandomForcedMatchmaker}]
      \item[\texttt{RandomAcceptanceMatchmaker}]
      \item[\texttt{RankBasedMatchmaker}]
      \item[\texttt{TournamentMatchmaker}]
      \item[\texttt{MultipleTournamentMatchmaker}]
    \end{description}
  \end{itemize}
\end{frame}

\begin{frame}
  \frametitle{\texttt{Mutagen}}
  \framesubtitle{\texttt{Int => Double}}
  \begin{itemize}
    \item determines the probability with which individuals mutate
    \item depends on the current generation
    \item implementations:
    \begin{description}
      \item[\texttt{ConstantMutagen}] $~$ \\ based on $f(x) = a$
      \item[\texttt{PolynomialMutagen}] $~$ \\ based on $f(x) = a + b \cdot x^n$
      \item[\texttt{ExponentialMutagen}] $~$ \\ based on $f(x) = a \cdot e^{b \cdot x}$
    \end{description}
  \end{itemize}
\end{frame}

% --------------------------------------------------------------------------------------------------
%  new section
% --------------------------------------------------------------------------------------------------

\section{Preparing the Benchmark}

\subsection{\texttt{EvolutionarySolver}}

\begin{frame}
  \frametitle{\texttt{EvolutionarySolver[A]}}
  \begin{itemize}
    \item genome type: \texttt{Vector[A]}
    \item problem type: \texttt{Vector[A] => Double}
    \item \texttt{Evolutionary} with
    \begin{description}
      \item[\texttt{def vars: Int}]
      \item[\texttt{def lower: Vector[Double]}]
      \item[\texttt{def upper: Vector[Double]}]
    \end{description}
    \item subclasses
    \begin{description}
      \item[\texttt{RealSolver extends EvolutionarySolver[Double]}]
      \item[\texttt{BinarySolver extends EvolutionarySolver[Boolean]}]
    \end{description}
  \end{itemize}
\end{frame}

\subsection{\texttt{SplitEvolver}}

\defverbatim[colored]\splitevolver{%
\begin{lstlisting}[basicstyle=\footnotesize,emph={def,implicit},emphstyle={\color{orange}}]
def apply[G,P]
  (ea: Evolutionary[G,P])
  (generations: Int = 500,
   individuals: Int = 100)
  (implicit matchmaker: Matchmaker = RankBasedMatchmaker,
            mutagen: Mutagen = ExponentialMutagen,
            debugger: Option[(Int,Double) => Unit] = None)
   : Individual[G]
\end{lstlisting}
}

\begin{frame}
  \frametitle{\texttt{SplitEvolver (I)}}
  \begin{itemize}
    \item partitions population
      \begin{itemize}
        \item first part mutates
        \item second part recombines
      \end{itemize}
    \item partition sizes depend on a \texttt{Mutagen}
  \end{itemize}
\end{frame}

\begin{frame}
  \frametitle{\texttt{SplitEvolver (II)}}
  \splitevolver
\end{frame}

\subsection{Execution}

\defverbatim[colored]\basicexecution{%
\begin{lstlisting}[basicstyle=\footnotesize,emph={val,new},emphstyle={\color{orange}}]
val f = Equation.ackley

val solver = new RealSolver(vars = 4, problem = f)

val solution = SplitEvolver(solver)
                           (generations = 2000)
                           (debugger = printer)
\end{lstlisting}
}

\begin{frame}
  \frametitle{Basic Execution}
  \basicexecution
\end{frame}

\defverbatim[colored]\chartexecution{%
\begin{lstlisting}[basicstyle=\footnotesize,emph={val,new},emphstyle={\color{orange}}]
val f = Equation.ackley

val solver = new RealSolver(vars = 4, problem = f)

val buf = collection.mutable.ListBuffer[(Int,Double)]()

SplitEvolver(solver)
            (generations = 2000)
            (debugger = charter(buf))

val chart = createLineChart (
  buf.toXYSeriesCollection("ackley")
)

chart.show()
\end{lstlisting}
}

\begin{frame}
  \frametitle{Chart Execution}
  \chartexecution
\end{frame}

% --------------------------------------------------------------------------------------------------
%  new section
% --------------------------------------------------------------------------------------------------

\section{The Benchmarks}

\fswframe{All Functions with Fittest Individual}{img/functions-fittest.pdf}
\fswframe{All Functions with Average Fitness}{img/functions-average.pdf}
\fswframe{Griewank \textendash\ Vector Size}{img/griewank-vector-sizes.pdf}
\fswframe{Griewank \textendash\ Population Size}{img/griewank-population-sizes.pdf}
\fswframe{Griewank \textendash\ Mutagen}{img/griewank-mutagens.pdf}
\fswframe{Griewank \textendash\ Crossover}{img/griewank-crossovers.pdf}
\fswframe{Griewank \textendash\ Matchmaker}{img/griewank-matchmakers.pdf}

% --------------------------------------------------------------------------------------------------
%  wrap it up
% --------------------------------------------------------------------------------------------------

\section*{}

%\begin{frame}
%  \frametitle{Conclusions}
%  \begin{enumerate}
%    \item 
%  \end{enumerate}
%\end{frame}

\frame[plain]{\begin{center}\Huge{EOL}\end{center}}

% --------------------------------------------------------------------------------------------------
%  appendix -- additional stuff for questions
% --------------------------------------------------------------------------------------------------

\appendix

\end{document}

